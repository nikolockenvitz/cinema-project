% !TEX root =  master.tex
\section{Ablauf im Kino}

Der von uns vorgesehene Ablauf im Kino lässt sich in 3 distinktive Arten unterteilen, die im Folgenden genauer beschrieben werden.


Zuerst wird ein Kinobesuch betrachtet, in welchem der Kunde keine Tickets im voraus gekauft oder reserviert hat. In diesem Falle würde der Kunde sich am Schalter über das aktuelle Kinoprogramm informieren und nach Auswahl eines Films, Vorstellungszeitpunkts und Sitzplatzes die Tickets kaufen. Nun kann der Kunde mit seinen vom Verkäufer erhaltenen Tickets das passende Kino betreten und den auf dem Ticket angegebenen Sitzplatz einnehmen. Die Ticketkontrolle findet dabei direkt vor dem Kinosaal statt, wo das Ticket vom Personal kontrolliert wird. Optional kann er davor natürlich noch Getränke und Snacks an der Snackbar kaufen. 
Zusammengefasst gibt es also vor Ort 3 Schritte, die die meisten Gäste durchlaufen: Ticketkauf am Schalter, Kauf von Snacks und Eintritt in den passenden Kinosaal.
Die Kinomitarbeiter müssen in diesem Fall einerseits das für den Kunden passende Ticket auswählen und verkaufen, andererseits aber auch die Snackbar besetzen und vor dem Kinosaal die Tickets überprüfen.


Etwas schneller läuft es ab, wenn der Kunde vorab seine Sitzplätze reserviert hat. In einem solchen Szenario fällt der erste Schritt in der Kette kürzer aus: Statt den Kunden zu beraten und Laufzeiten heraussuchen zu müssen, kann der Kinomitarbeiter am Schalter einfach per Reservierungsnummer oder Name die Tickets für die vorab gewählten Sitzplätze verkaufen. Dies verkürzt allgemein die Bearbeitungszeit für einzelne Kunden, was zu kürzeren Wartezeiten am Schalter führt. Der sonstige Ablauf bleibt jedoch weitestgehend gleich.


Der schnellste und einfachste Ablauf vor Ort bietet sich dann, wenn der Kunde bereits im voraus seine Tickets online erworben hat. In diesem Fall wird der Schritt am Schalter obsolet; Stattdessen können die Kunden direkt Snacks erwerben oder zum Kinosaal gehen, wo ihre entweder im voraus ausgedruckten oder auf dem Handy gespeicherten Online-Tickets kontrolliert werden. Bis auf den Vorgang am Schalter bleibt auch hier der sonstige Ablauf gleich, weshalb dieser Ablauftyp für sowohl Kinopersonal als auch Kunden am unkompliziertesten ist.