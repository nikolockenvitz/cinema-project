% !TEX root =  master.tex
\section{User-Stories}
\chaptermulitpleauthor{\authorGP}{\authorRF, \authorEJ}

Mit der großen Spannbreite an Personas gibt es auch zahlreiche, teils sich stark unterscheidende Anforderungen an das Kinobuchungssystem.
Im folgenden Abschnitt werden diese Wünsche der Personas aufgezählt und gruppiert.

\subsection{Übersicht / Filmauswahl}
Zuerst werden die Anforderungen an eine Übersicht über das aktuelle Filmprogramm geschildert.

\cvsect{Florentina Kastenkette}
Als Florentina Kastenkette interessiere ich mich primär nur für eine schnelle Übersicht der neuesten und beliebtesten Filme.

\cvsect{Leon Schweickert}
Als Leon Schweickert finde ich eine ausführliche Auflistung aller Filme geeigneter, um mich über alle Filme ganeuer informieren zu wollen und somit die Besetzung und Vorschau eines Films ansehen zu können oder auch die Beschreibung des Films lesen zu können.
\\
Als Leon Schweickert würde würde ich mich für einen E-Mail-Newsletter mit dem Kinoprogramm interessieren, um immer auf dem Laufenden zu bleiben.

\cvsect{Familie Mandick}
Als Familie Mandick legen wir einen großen Wert darauf, dass wir das vielfältige Filmprogramm nach Genre oder \acs{FSK}-Bewertung filtern können, um auch für unsere Kinder geeignete Filme zu finden.
\\
Als Familie Mandick möchten wir eine vereinfachte Suche nach Filmen, um auch spontan das Kino besuchen zu können.

\cvsect{Oma Gertrud}
Als Oma Gertrud möchte ich Werbung zu neuen Kinofilmen als Zeitungsan­non­ce sehen, um immer auf dem neuesten Stand zu sein.

\subsection{Auswahl einer Vorstellung}
Nach der Auswahl eines Films folgt die Wahl einer Vorstellung (Zeit und Datum).

\cvsect{Johnny Cash}
Als Johnny Cash möchte ich eine klare Übersicht der nächsten Vorstellungen, um so schnell wie möglich den passenden Termin für die Kunden zu finden.

Als Johnny Cash möchte ich nach zukünftigen Vorstellungen suchen und filtern können, um den Kunden bei der Reservierung dieser zu helfen.

\cvsect{Leon Schweickert}
Als Leon Schweickert möchte ich bereits auf der Übersicht die nächsten Vorstellungen zu den Blockbustern sehen, um Zeit beim Buchungsvorgang zu sparen.

\cvsect{Florentina Kastenkette}
Als Florentina Kastenkette möchte ich eine Detailseite der Filme mit allen Vorstellungen, um mit meinen Freundinnen den Kinobesuch perfekt planen zu können.
\\
Als Florentina Kastenkette möchte ich schnell erkennen oder filtern können, welche Vorstellungen in 2D bzw. 3D sind, und ob die Filme in deutscher Sprache oder in ihrer Originalfassung aufgeführt werden, um nicht beim Kinobesuch enttäuscht zu werden.

\cvsect{Oma Gertrud}
Als Oma Gertrud möchte ich im Kino anrufen und mir Vorstellungen von einem Mitarbeiter ansagen und anschließend einen Platz auf meinen Namen reservieren lassen.

\subsection{Sitzplatzauswahl}
\cvsect{Florentina Kastenkette}
Als Florentina Kastenkette möchte ich mehrere Plätze mit unterschiedlichen Ermäßigungsstufen auswählen zu können, um auch mal mit meiner minderjährigen Schwester ins Kino gehen zu können.

\cvsect{Leon Schweickert}
Als Leon Schweickert möchte ich eine hübsche grafische Platzauswahl, bei der man auf einen Blick zwischen belegten und unbelegten Plätzen unterscheiden kann, sodass ich beim Buchen keine Zeit mit der Platzsuche verschwenden muss.

\cvsect{Johnny Cash}
Als Johnny Cash möchte ich, dass Reservierungen und Buchungen von mir eine höhere Priorität haben als die der Online-Nutzer, um den Kunden vor Ort ein angenehmes Kinoerlebnis zu ermöglichen.

\subsection{Bezahlung}
\cvsect{Oma Gertrud}
Als Oma Gertrud möchte meine Tickets nur reservieren und später im Kino bar bezahlen, um nicht meine Kreditkartendaten im Internet freizugeben.

\cvsect{Florentina Kastenkette}
Als Florentina Kastenkette möchte ich vor Ort bezahlen können, um meine Ausgaben auf der Kreditkartenabrechnung sehen zu können.

\cvsect{Leon Schweickert}
Als Leon Schweickert möchte ich online per PayPal bezahlen und dies auch in meinem Nutzerkonto als bevorzugtes Zahlungsmittel hinterlegen können, um nicht bei jeder Buchung meine Daten neu einzugeben.

\cvsect{Johnny Cash}
Als Johnny Cash möchte ich eine möglichst große Auswahl an Bezahlungsmöglichkeiten anbieten, um den Bezahlvorgang so unproblematisch wie möglich zu gestalten.
\\
Als Johnny Cash möchte ich die Gesamtpreise berechnet bekommen, um so schnell den Preis dem Kunden sagen zu können.

\subsection{Bestätigung}
\cvsect{Leon Schweickert}
Als Leon Schweickert möchte ich neben einer allgemeinen Bestätigung auf der Webseite, diese auch auf meinem Smartphone erhalten, um auch unterwegs mich für einen Kinobesuch entscheiden zu können.

\cvsect{Florentina Kastenkette}
Als Florentina Kastenkette möchte ich meine Buchung direkt auf meinen sozialen Netzwerken teilen, um meine Freunde und Follower immer auf dem neusten Stand zu halten.

\subsection{Reservierungsbearbeitung}
\cvsect{Familie Mandick}
Als Familie Mandick möchten wir unsere gebuchten Tickets auch noch spontan kündigen können, da bei uns schnell mal etwas dazwischen kommt. 

\cvsect{Leon Schweickert}
Als Leon möchte ich meine Reservierungen oder gekauften Tickets per Knopfdruck in meinen (mobilen) Kalender übertragen können, wobei ich gern auch noch eine eigene Notiz anhängen können würde.

\subsection{Im Kino}
\cvsect{Oma Gertrud}
Als Oma Getrud möchte ich mein Ticket an der Kasse erhalten, dies ist für mich als Oma Gertrud sehr wichtig, da ich gern ein Stück Papier vorzeigen können möchte.

\cvsect{Johnny Cash}
Als Johnny ist es mir wichtig, schnell auf online reservierte Tickets an der Kasse zugreifen zu können.

\cvsect{Familie Mandick \& Leon Schweickert}
Als Familie Mandick oder auch Leon Schweickert wollen wir den Schritt an der Kasse überspringen und direkt den Weg Richtung Kinosaal antreten.
Als Leon möchte ich die gesparte Zeit durch die Onlinebuchung verwenden um Snacks für das Kino zu kaufen.
Als Familie Mandick möchte wir nur eine kurze Zeit in der Warteschlange verbringen um die Kinder vor dem Kinoerlebnis nicht mit unnötiger Warterei zu foltern.
Ebenfalls möchten wir die Möglichkeit haben um so spät wie möglich zu einem Kinofilm anzukommen.

\cvsect{Florentina Kastenkette \& Leon Schweickert}
Als einer der beiden jungen Nutzer (Florentina oder Leon) möchte ich meinen Studentenausweis bzw. Schülerausweis in meinem Benutzerprofil hinterlegen, um nicht bei jedem Kinobesuch den Studentenausweis vorzeigen zu müssen.

\subsection{Sonstiges}
Die sonstigen Wünsche beziehen sich hauptsächlich auf häufige Besucher und Webseiten-Nutzer wie Leon.

Das Abspeichern von Filterpräferenzen, eine Einsicht der eigenen Kinohistorie und -statistik oder das Erstellen von Bewertungen und Kommentaren für die besuchten Filme wären Features, die die Webseite interaktiv für den Anwender gestalten würden.

Außerdem wäre es für mich als Leon nützlich, von Rabattaktionen oder Bonusprogrammen profitieren zu können, da ich so oder so viel Zeit im Kino verbringe.
