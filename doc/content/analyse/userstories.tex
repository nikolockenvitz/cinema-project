% !TEX root =  master.tex
\section{User-Stories}

Mit der großen Spannbreite an Personas gibt auch zahlreiche, teils sich stark unterscheidende Anforderungen an das Kinobuchungssystem. Im folgenden Abschnitt werden diese Wünsche der Personas aufgezählt und gruppiert.

\subsection{Übersicht / Filmauswahl}
Zuerst werden die Anforderungen an eine Übersicht über das aktuelle Filmprogramm geschildert. Als Florentina Kastenkette interessiere ich mich primär nur für eine schnelle Übersicht über die neuesten und beliebtesten Filme, während für mich ich als Leon Schweickert eine ausführliche Auflistung aller Filme geeigneter ist. Hier könnte ich mich über alle Filme genauer informieren, mir Besetzung oder eine Vorschau ansehen und sogar eine Kurzbeschreibung des Films durchlesen. Ich würde mich eventuell sogar für einen Newsletter mit dem Kinoprogramm interessieren, den ich einfach per Mail erhalten könnte. Als Familie Mandick legen wir einen größeren Wert darauf, das oftmals große Filmprogramm nach Genre oder FSK-Bewertung filtern zu können. Hierdurch ließen sich für die ganze Familie geeignete Filme schnell finden, was auch durch eine Suchfunktion erreicht werden könnte. Dies würde uns trotz unserer spontanen Natur einen kurzfristigen Kinobesuch erleichtern. Als alte Oma Gertrud ist all das jedoch kaum relevant, da ich mit einem dieser Computer kaum zurecht komme, was ich ja auch nicht lernen muss, da sich dieses Internet eh nicht lange halten wird. Für mich wäre eine ausgedruckte Version der Übersicht in Form einer Zeitungsanzeige oder eines Flyers wesentlich zugänglicher.

\subsection{Auswahl einer Vorstellung}
Nach der Auswahl eines Films folgt die Wahl einer Vorstellung (Zeit und Datum). Hier ist es für mich als Johnny Cash sehr wichtig, eine klare Übersicht über die nächsten Vorstellungen zu bekommen, um so schnell wie möglich den passenden Termin für die Kunden zu finden. Hierbei sind für mich vor allem die zeitnahen Vorstellungen relevant, da die meisten Kunden am Schalter direkt Tickets für die Vorstellungen am selben Abend kaufen. Bei Fragen nach zukünftigen Filmvorstellungen würde ich zwar gern nach solchen Filmen filtern können, an sich treten solche Fälle jedoch wesentlich seltener auf. Auch für mich als Leon Schweickert spielt Geschwindigkeit eine große Rolle: Ich hätte am liebsten schon bei der Übersicht die Vorstellungszeiten am aktuellen Tag, damit ich mir einen Klick sparen zu kann. Eine eigene Seite zur Vorstellungsauswahl wird jedoch von  mir als Florentina Kastenkette bevorzugt, da ich mit meinen Freundinnen am Telefon gemeinsam aushandeln muss, wann alle Zeit haben, und dabei eine große Übersicht echt praktisch wäre. Außerdem möchte ich schnell erkennen oder Filtern können, welche Vorstellungen in 2- oder 3D sind, und ob die Filme in deutscher Sprache oder in ihrer Originalfassung aufgeführt werden. Als Oma Gertrud möchte ich lieber im Kino anrufen und mir eine Vorstellungsauswahl von einem Mitarbeiter ansagen und anschließend einen Platz auf meinen Namen reservieren lassen.

\subsection{Sitzplatzauswahl}
Bei der Platzauswahl ist es mir als Florentina wichtig, mehrere Plätze mit unterschiedlichen Ermäßigungsstufen auswählen zu können, da ich neben meinen Freundinnen auch gern meine kleine Schwester mit ins Kino nehme. Als Leon hingegen erwartet ich eine hübsche grafische Platzauswahl, bei der man auf einen Blick zwischen belegten und unbelegten Plätzen unterscheiden kann. Für mich als Kassierer Johnny ist es wichtig, dass die von mir getätigten Reservierungen und Buchungen vor denen der Online-Nutzer Vorrang haben, damit mir keine Plätze beim Bedienen der Kunden vor Ort "weggeschnappt" werden. Dies würde einerseits meinen täglichen Ablauf vor Ort erschweren und mich andererseits unprofessionell wirken lassen.

\subsection{Bezahlung}
Ich als Oma Gertrud möchte meine Tickets nur reservieren und später im Kino bar bezahlen, da ich selbst nicht weiß, wie man Geld mit dem Computer verschicken könnte. Auch als Florentina bezahle ich lieber vor Ort, dies liegt jedoch daran, dass ich überall mit meiner EC-Karte bezahle, damit ich meine Kosten immer im Blick behalten kann und es meistens schneller geht als mein Bargeld herauszukramen. Falls ich jedoch mal online bezahle, möchte ich mich aber nicht mit einem Konto anmelden müssen, da ich "nur" sozialen Netzwerken meine Nutzerdaten anvertraue. Ganz anders geht es mir hier als Leon: Ich bevorzuge es, online per Paypal zu bezahlen und möchte am liebsten bei meinem Nutzerkonto ein bevorzugtes Zahlungsmittel hinterlegen können, um dies nicht bei jeder Buchung erneut eintragen zu müssen. Als Johnny Cash ist es mir an der Kasse wichtig, eine möglichst große Anzahl an Bezahlungsmitteln anbieten zu können, um den Bezahlvorgang so unproblematisch wie möglich gestalten zu können. Außerdem möchte ich keine Gesamtpreise selbst errechnen müssen, sondern schnell den Preis für bereits reservierte Tickets herausfinden können, ohne groß danach suchen zu müssen.

\subsection{Bestätigung}
Neben einer allgemeinen Bestätigung auf der Webseite, dass die Bezahlung, Buchung oder Reservierung erfolgreich gewesen ist, möchte ich als Leon nach meiner Buchung das Ticket direkt auf meinem Handy zur Verfügung stehen haben. Ob das per App oder Mailversand passiert, ist mir dabei eher unwichtig, ich möchte bloß "unnötigen Papierkram" umgehen. Da ich als Florentin sehr aktiv auf sozialen Netzwerken unterwegs ist, wäre es mir wünschenswert, eine "Teilen"-Funktionalität auf der Bestätigungsseite zu haben, damit ich direkt online mit all meinen Freunden teilen kann, wann und wo der Kinobesuch stattfindet. Dies würde mir ein lästiges erneutes Eintippen der Daten ersparen.

\subsection{Reservierungsbearbeitung}
Die Möglichkeit, getätigte Reservierungen online oder per Anruf spontan kündigen zu können, spielt für unsere Familie Mandick eine große Rolle. Hiermit können wir sicher gehen, dass wir nicht unnötig Geld für die Tickets verschwenden, obwohl doch vielleicht wieder etwas dazwischen kommt. Als Leon möchte ich meine Reservierungen oder gekauften Tickets per Knopfdruck in meinen (mobilen) Kalender übertragen können, wobei ich gern auch noch eine eigene Notiz anhängen können würde. 

\subsection{Im Kino}
Ein ausgedrucktes Ticket an der Kasse zu erhalten ist für mich als Oma Gertrud sehr wichtig, da ich gern "etwas festes in der Hand" habe und nicht nur "so eine komische Würfelgrafik". Als Johnny ist es mir dementsprechend wichtig, online oder am Telefon reservierte Tickets an der Kasse schnell ausdrucken zu können. Die Familie Mandick und Leon Schweickert wollen den Schritt an der Kasse lieber überspringen; Als Leon bevorzuge ich es, die gesparte Zeit zum Popcorn-Kauf aufzuwenden, während wir, die Eltern Mandick, bloß ein rumgequengele unserer Kinder in der Warteschlange verhindern wollen. Außerdem wäre ein schneller Weg zum Kinosaal für uns sehr nützlich, wenn wir mal wieder "punktgenau" zum Filmstart im Kino ankommen und uns beeilen müssen, um nicht den Filmstart zu verpassen. Als einer der beiden eher jungen Nutzer Florentina oder Leon wäre es außerdem praktisch, wenn ich nicht jedes Mal im Kino den Studentenausweis vorzeigen müsste, sondern ich diesen entweder bei der Buchung angeben oder im Profil hinterlegen könnte, falls man ihn mal vergisst.

\subsection{Sonstiges}]
Die sonstigen Wünsche beziehen sich hauptsächlich auf häufige Besucher und Webseiten-Nutzer wie Leon; Das Abspeichern von Filterpräferenzen, eine Einsicht der eigenen Kinohistorie und -Statistik oder das erstellen von Bewertungen und Kommentaren für die besuchten Filme wären Features, die die Webseite interaktiv für den Anwender gestalten würde. Außerdem wäre es für mich als Leon nützlich, von Rabattaktionen oder Bonusprogrammen profitieren zu können, da ich so oder so viel Zeit im Kino verbringe.
