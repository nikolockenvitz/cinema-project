% !TEX root =  master.tex
\section{User-Stories}

Mit der großen Spannbreite an Personas gibt es auch zahlreiche, teils sich stark unterscheidende Anforderungen an das Kinobuchungssystem.
Im folgenden Abschnitt werden diese Wünsche der Personas aufgezählt und gruppiert.

\subsection{Übersicht / Filmauswahl}
Zuerst werden die Anforderungen an eine Übersicht über das aktuelle Filmprogramm geschildert.
Während sich Florentina Kastenkette primär nur für eine schnelle Übersicht über die neuesten und beliebtesten Filme interessiert, ist für Leon Schweickert eine ausführliche Auflistung aller Filme geeignet.
Hier könnte er sich über alle Filme genauer informieren, sich Besetzung oder eine Vorschau ansehen und sogar eine Kurzbeschreibung des Films durchlesen.
Für ihn wäre eventuell sogar ein Newsletter mit dem Kinoprogramm geeignet, den er per Mail erhalten könnte.
Die Familie Mandick hingegen legt einen größeren Wert darauf, das oftmals große Filmprogramm nach Genre oder FSK-Bewertung filtern zu können.
Hierdurch ließen sich für die ganze Familie geeignete Filme schnell finden, was auch durch eine Suchfunktion erreicht werden könnte.
Dies würde der Familie trotz ihrer spontanen Natur einen kurzfristigen Kinobesuch erleichtern.
Für Oma Gertrud ist all das jedoch kaum relevant, da sie mit einem Computer kaum zurecht kommt und denkt, dass sich das Internet nicht lange halten wird.
Für sie wäre eine ausgedruckte Version der Übersicht in Form einer Zeitungsanzeige oder eines Flyers wesentlich zugänglicher.

\subsection{Auswahl einer Vorstellung}
Nach der Auswahl eines Films folgt die Wahl einer Vorstellung (Zeit und Datum).
Hier ist es für Johnny Cash sehr wichtig, eine klare Übersicht über die nächsten Vorstellungen zu bekommen, um so schnell wie möglich den passenden Termin für die Kunden zu finden.
Für ihn sind vor allem die zeitnahen Vorstellungen relevant, da die meisten Kunden am Schalter direkt Tickets für die Vorstellungen am selben Abend kaufen.
Bei Fragen nach zukünftigen Filmvorstellungen würde er zwar gern nach solchen Filmen filtern können, an sich treten solche Fälle jedoch wesentlich seltener auf.
Auch für Leon Schweickert spielt Geschwindigkeit eine große Rolle: Er hätte am liebsten schon bei der Übersicht die Vorstellungszeiten am aktuellen Tag, um sich einen Klick sparen zu können.
Eine eigene Seite zur Vorstellungsauswahl wird jedoch von Florentina Kastenkette bevorzugt, da sie mit ihren Freundinnen am Telefon gemeinsam aushandeln muss, wann alle Zeit haben, und dabei eine große Übersicht gut hilft.
Außerdem möchte Sie schnell erkennen oder filtern können, welche Vorstellungen in 2- oder 3D sind, und ob die Filme in deutscher Sprache oder in ihrer Originalfassung aufgeführt werden.
Oma Gertrud möchte lieber im Kino anrufen und sich eine Vorstellungsauswahl von einem Mitarbeiter ansagen und anschließend einen Platz auf ihren Namen reservieren lassen.

\subsection{Sitzplatzauswahl}
Bei der Platzauswahl ist es Florentina wichtig, mehrere Plätze mit unterschiedlichen Ermäßigungsstufen auswählen zu können, da sie neben ihren Freundinnen auch gern ihre kleine Schwester mit ins Kino nimmt.
Leon hingegen erwartet eine hübsche grafische Platzauswahl, bei der man auf einen Blick zwischen belegten und unbelegten Plätzen unterscheiden kann.
Als Kassierer ist es Johnny wichtig, dass die von ihm getätigten Reservierungen und Buchungen vor denen der Online-Nutzer Vorrang haben, damit ihm keine Plätze beim Bedienen der Kunden vor Ort "weggeschnappt" werden.
Dies würde einerseits den Ablauf vor Ort erschweren und ihn andererseits unprofessionell wirken lassen.

\subsection{Bezahlung}
Oma Gertrud möchte ihr Ticket nur reservieren und im Kino bar bezahlen, da sie selbst nicht weiß, wie man Geld mit dem Computer verschickt.
Auch Florentina bezahlt lieber vor Ort, dies liegt jedoch daran, dass sie überall mit ihrer EC-Karte bezahlt, um ihre Kosten immer im Blick zu behalten und es meistens schneller geht als Bargeld herauszukramen.
Falls sie doch online bezahlt, möchte sie sich jedoch nicht mit einem Konto anmelden müssen, da sie "nur" sozialen Netzwerken ihre Nutzerdaten anvertraut.
Ganz anders geht es hier Leon: Er bevorzugt es, online per PayPal zu bezahlen und möchte am liebsten bei einem Nutzerkonto sein bevorzugtes Zahlungsmittel hinterlegen können, um dies nicht bei jeder Buchung erneut eintragen zu müssen.
Für Johnny Cash ist es an der Kasse wichtig, eine möglichst große Anzahl an Bezahlungsmitteln anbieten zu können, um den Bezahlvorgang so unproblematisch wie möglich gestalten zu können.
Hinzu kommt, dass er keine Gesamtpreise selbst errechnen und schnell den Preis für bereits reservierte Tickets herausfinden möchte, ohne groß danach suchen zu müssen.

\subsection{Bestätigung}
Neben einer allgemeinen Bestätigung auf der Webseite, dass die Bezahlung, Buchung oder Reservierung erfolgreich gewesen ist, möchte Leon nach seiner Buchung das Ticket direkt auf seinem Handy zur Verfügung stehen haben.
Ob das per App oder Mailversand passiert, ist ihm dabei eher unwichtig, er möchte bloß "unnötigen Papierkram" umgehen.
Da Florentin sehr aktiv auf sozialen Netzwerken unterwegs ist, wäre es ihrerseits wünschenswert, eine "Teilen"-Funktionalität auf der Bestätigungsseite zu haben, damit sie direkt online mit all ihren Freunden teilen kann, wann und wo ihr Kinobesuch stattfindet.
Dies würde ihr ein lästiges erneutes Eintippen der Daten ersparen.

\subsection{Reservierungsbearbeitung}
Die Möglichkeit, getätigte Reservierungen online oder per Anruf spontan kündigen zu können, spielt für die Familie Mandick eine große Rolle.
Hiermit können Sie sicher gehen, dass sie nicht unnötig Geld für die Tickets verschwenden, obwohl doch vielleicht wieder etwas dazwischen kommt.
Leon möchte seine Reservierungen oder gekauften Tickets per Knopfdruck in seinem (mobilen) Kalender übertragen können, wobei er gern auch noch eine eigene Notiz anhängen können würde.

\subsection{Im Kino}
Ein ausgedrucktes Ticket an der Kasse zu erhalten ist für Oma Gertrud sehr wichtig, da sie gern "etwas Festes in der Hand hat" und nicht nur "so eine komische Würfelgrafik".
Johnny ist es dementsprechend wichtig, online oder am Telefon reservierte Tickets an der Kasse schnell ausdrucken zu können.
Die Familie Mandick und Leon Schweickert wollen den Schritt an der Kasse lieber überspringen.
Leon bevorzugt es, die gesparte Zeit zum Popcorn-Kauf aufzuwenden, während die Eltern Mandick bloß verhindern wollen, dass ihre Kinder anfangen, in der Schlange rumzuquengeln und laut zu werden.
Außerdem wäre ein schneller Weg zum Kinosaal für die Mandicks sehr nützlich, wenn sie mal wieder "punktgenau" zum Filmstart im Kino ankommen und sich beeilen müssen, um nicht den Anfang zu verpassen.
Für die beiden eher jungen Florentina und Leon wäre es außerdem praktisch, nicht jedes Mal im Kino den Studentenausweis vorzeigen zu müssen, sondern diesen entweder bei der Buchung angeben oder im Profil hinterlegen zu können, falls man ihn mal vergisst.

\subsection{Sonstiges}
Die sonstigen Wünsche beziehen sich hauptsächlich auf häufige Besucher und Webseiten-Nutzer wie Leon.
Das Abspeichern von Filterpräferenzen, eine Einsicht der eigenen Kinohistorie und -statistik oder das Erstellen von Bewertungen und Kommentaren für die besuchten Filme wären Features, die die Webseite interaktiv für den Anwender gestalten würden.
Außerdem wäre es für Leon nützlich, von Rabattaktionen oder Bonusprogrammen profitieren zu können, da er so oder so viel Zeit im Kino verbringt.
