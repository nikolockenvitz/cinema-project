% !TEX root =  master.tex
\section{User-Journey}

Anhand der nun weitestgehend geschilderten Anforderungen aller Personas wurde der Aufbau der Website für das entwickelte Kinobuchungssystem entschieden.
Diese gewählte Struktur wird im Folgenden anhand eines Beispiels geschildert, in welchem ein Endnutzer drei Tickets für eine spezielle Filmvorführung bucht.
Vorrausgesetzt ist hierbei, dass der Nutzer tatsächlich eine Online-Buchung vornimmt und nicht direkt im Kino anruft, um ein Ticket zu erwerben/ einen Sitzplatz zu reservieren.

Bei Aufruf der Kino-Hauptseite wird ein Nutzer mit einer Übersicht über alle momentan im Kino laufenden Filme präsentiert. Durch herauf- und herabscrollen lässt sich schnell durch die Liste navigieren, das finden des gesuchten Films wird durch die großen Cover-Bilder neben dem Titel erleichtert. Durch Anwählen des Filmtitels wird der Nutzer auf die Film-eigene Seite geleitet, auf der alle Vorstellungen des Films innerhalb der nächsten Tage aufgelistet sind. Durch eine erneute Auswahl des Titels oder Titelbilds gelangt man auf eine weitere Seite mit einer detaillierten Beschreibung und Bewertung des Films; In dem gewählten Anwendungsfall ist jedoch die Anwahl einer Vorstellung in der Übersicht zielführend. Eine jede dieser Vorstellungen ist durch ein Icon in der nach Tagen geordneten Tabelle repräsentiert, mit der passenden Uhrzeit als Beschriftung. Wurde das Icon mit dem gewünschten Vorstellungszeitpunkts ausgewählt, so wird der Nutzer zur Sitzplatzübersicht und -auswahl weitergeleitet.

Nun hat der Nutzer sowohl den gewünschten Film sowie Vorstellungszeitpunkt gewählt. Als nächstes ist für den Anwender die Wahl der Ticketanzahl und damit verbundenen Sitzplätze relevant. Dies geschieht im entwickelten Buchungssystem durch die Anwahl der Plätze auf einer 2D-Grafik des Kinosaals, in welcher bereits belegte, freie und eigens angewählte Plätze farblich voneinander abgehoben sind. Nach der Auswahl der Position und Anzahl der gewünschten Sitzplätze sollte der Nutzer nun in der Lage sein, die Preiskategorien der Plätze (Ermäßigungen usw.) auszuwählen und den Einzel- sowie Gesamtpreis der Auswahl einsehen zu können.

Ist der Nutzer mit diesem Schritt fertig, kommt es nun zum Bezahlungsvorgang für die ausgewählten Tickets. Hier soll der Nutzer eine kompakte Übersicht über die Bestellung erhalten, und nach Überprüfung seiner Auswahl seine persönlichen Daten (Vorname, Nachname und E-Mail-Adresse) angeben können. Hinzu wird noch die Auswahl zwischen mehreren Zahlungsmethoden, sowie die Eingabemöglichkeit für Rabattcodes geboten. Um die Bezahlung abschließen zu können, liest der Anwender dann die allgemeinen Geschäftsbedingungen und bestätigt anschließend diese und die Bezahlung.

Mit der Bezahlbestätigung kommt der Nutzer zum letzten Schritt des Buchungsvorgangs, der Bestellbestätigung. Hier werden erneut die zusammengefassten Details der Bestellung (Kosten, Sitzplätze, Filmdetails und Vorstellungszeitpunkt) sowie ein Bestätigungstext dargestellt. Außerdem wird dem Nutzer sein digitales Ticket in Form eines QR-Codes angezeigt, welches der Anwender auch per Mail zugesandt bekommt oder mithilfe eines Accounts von der Mobile-App abrufen kann. Dieser Code enthält alle Infos über das Ticket und ist im Kino gleichwertig mit einer ausgedruckten Version.