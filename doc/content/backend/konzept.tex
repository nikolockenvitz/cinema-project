\section{Konzept}
\label{sec:konzept}
In diesem Projekt soll die Drei-Schichten-Architektur eingesetzt werden.
Sie dient dazu, die einzelnen Schichten zu separieren, sodass in zukünftigen Schritten ein Austausch der genutzten Benutzeroberfläche oder Datenbank möglich ist. 

Wie zuvor in Kapitel \vref{sec:technologien} beschrieben, finden in diesem Projekt \acs{REST}ful Webservices Anwendung.
Da Webservices in der Realität u.a. aus Performance-Gründen auf unterschiedlichen Servern ausgeführt werden, wurde dies hier ebenfalls umgesetzt.
Zum Einsatz kommt eine System- und Data-Ressource.
Wie in Abbildung \vref{fig:konzept_backend} dargestellt ist, greift die Benutzeroberfläche niemals direkt auf die Datenhaltungsschicht.
Dies wird realisiert, indem das Front-End auf die bereitgestellten Webservices in der Fachkonzeptschicht zugreift.
Diese wiederum greifen dann auf die Datenhaltungsschicht zu.

\begin{figure}[ht]
	\centering
	\includegraphics[width=0.5\textwidth]{img/backend/drei-schichten-architektur}
	\captionsetup{format=hang}
	\caption{Konzept des Back-Ends \\ eigene Darstellung mittels \url{https://draw.io/}}
	\label{fig:konzept_backend}
	\end{figure}

Die System-Ressource stellt dem Front-End mehrere Webservices zur Verfügung.
Nachdem die Ressource über eine \acs{URI} \url{http://localhost:8080/cinema-system/show/1} aufgerufen wurde, ruft diese ihren eigenen Service \textit{getMovieById} auf, der den Webservice \url{http://localhost:8080/cinema-data/show/1} in der Data-Ressource aufruft, welche die gewünschten Daten in der Datenbank anfragt und in ein Transferobjekt umwandelt.
Eine Erläuterung warum dies notwendig ist, wird in Kapitel \vref{sec:dto} näher beschrieben. \\
Jede Ressource hat wiederum mehrere Services implementiert, die die gewünschten Ergebnisse konsolidieren.

Die Erläuterung wie dies seitens des Front-Ends umgesetzt ist, wird in Kapitel \vref{sec:anbindung_backend} näher erläutert.
 
%Möchte man z.B. alle Daten einer Vorstellung haben, so ruft das Front-End die Ressource mit der \acs{URI} \url{http://localhost:8080/cinema-system/show/1} auf.

Die aktuell verwendeten Ressourcen in diesem Projekt sind:
\begin{itemize}
	\item show $\rightarrow$ hier können alle Informationen über die Vorstellungen abgerufen werden
	\item reservation $\rightarrow$ alles was mit der Reservierung bzw. Blocken in Abhängigkeit steht 
	\item movie $\rightarrow$ hier können alle Informationen über eine Film abgerufen werden
	\item employee $\rightarrow$ hier können alle Informationen über die Mitarbeiter des Kinos abgerufen werden
\end{itemize} 