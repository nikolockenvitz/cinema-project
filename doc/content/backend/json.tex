\section{JavaScript Object Notation (JSON)}
\label{sec:json}
Für den Datenaustausch zwischen Front- und Backend wird mittels der \ac{JSON} realisiert. Grund ist, dass hier rein die Nutzdaten transferiert werden. Ferner gestalte sich das Auslesen und Generieren der \ac{JSON}-Dateien als recht \enquote{einfach}. \\
Ein Array bzw. Liste wird in \ac{JSON} mit [ ] dargestellt. Objekte werden mit \{ \} dargestellt. Jedes Attribut und jede Zeichenkette wird in \ac{JSON} in Anführungszeichen gesetzt (siehe Quelltext \vref{lst:example_movie}).  

\begin{minipage}{\linewidth}
\begin{lstlisting}[language=json,firstnumber=1]
{"movie": {
	"id": 1,
	"title": "Kampf der Titanen",
	"duration": 120,
	"hall":
	{
	"id": 1,
	"name": "Kino 1,
	"seats": [{"seat":{"id":1, ...}}]
	}, 
	"shows": [{"show":{"id":1, ...}}]
	}
}
\end{lstlisting}
\captionof{lstlisting}{Beispiel eines Films im JSON-Format}
\label{lst:example_movie}
\end{minipage}