\section{Representational State Transfer (REST)}
\label{sec:rest}
Um die Dreischichtenarchitektur zu etablieren, wird für den Datenaustausch zwischen Server--Frontend und Server--Datenbank auf das  \ac{REST} gesetzt. \ac{REST} ist ein Architekturstil für den Entwurf verteilter Netz"-werk"-an"-wen"-dung"-en. Es setzt auf \ac{HTTP} und \ac{HTTPS}-Methoden und stellt verschiedene Methoden zur Verfügung. Für den Einsatz von \ac{REST}-Services kommen folgende CRUD-Prinzipien in Frage:

\begin{table}
\begin{tabular}{ l| l | l }
	CRUD & HTTP & Beispiel-URI \\
	\hline
	Create & POST & http://lolcalhost:8080/cinema-system/movie \\ 
	Read & GET & http://lolcalhost:8080/cinema-system/show/1 \\ 
	Update & PUT & http://lolcalhost:8080/cinema-system/movie/5\\ 
	Delete & DELETE & http://lolcalhost:8080/cinema-system/customer/4711\\ 
\end{tabular}
\caption{CRUD-Befehle und deren Verwendung}
\label{tab:crud}
\end{table} 

Wir der Webservice mit einem \textit{POST} aufgerufen wird ein neuer Datensatz in der Datenbank angelegt. In dem Beispiel aus Tabelle \vref{tab:crud} wird ein neuer Film in der Datenbank angelegt. Wird die Ressource mit einem \textit{GET} aufgerufen wird ein Datensatz bei der Datenbank angefragt und an die Ressource zurückgesendet. In dem Beispiel aus Tabelle \vref{tab:crud} wird die Ressource Film aufgerufen und der Datenbankeintrag mit der Film-ID 1 abgefragt. Mittels \textit{PUT} wird ein vorhandener Datensatz aktualisiert. In dem Beispiel aus Tabelle \vref{tab:crud} würde z.B. ein neuer Schauspieler hinzugefügt werden. Schlussendlich löscht man einen Datensatz indem man die Ressource mit \textit{DELETE} aufruft. In dem Beispiel aus Tabelle  \vref{tab:crud} würde man den Kunden mit der Kundennummer 4711 aus der Datenbank entfernen.  

Während Anfragen über die REST-Schnittstelle über den Server beantwortet werden, können auch verschiedene HTTP bzw. HTTPS-Satuse zurückgeliefert werden (siehe Tabelle \vref{tab:http_status}). 

\begin{table}
\begin{tabular}{l | p{9.50 cm}}
	Satus Code & Beschreibung \\
\hline
200 (OK) & Alles OK bei der Verbindung \\
400 (Bad Request) & Der Server weiß nicht, was er mit der Anfrage machen soll \\
401 (Unauthorized) & Der Clien muss sich erst authentifizieren z.B. falsches Passwort gewählt \\
501 (Media Type unsupported) & Der gewählte Medientyp wird nicht unterstützt z.B. Text Plain oder es wird ein anderes Format erwartet z.B. JSON\\
503 (Service Unavailable) & Anfrage konnte nicht abgeschlossen werden 
\end{tabular}
\caption{Mögliche HTTP-Satuse eines Webservice}
\label{tab:http_status}
\end{table}