\section{Representational State Transfer (REST)}
\label{sec:rest}
\authorsection{\authorSG}
Um die Dreischichtenarchitektur zu etablieren, wird für den Datenaustausch zwischen Server--Front-End und Server--Datenbank auf das \acf{REST} gesetzt.
\acs{REST} ist ein Architekturstil für den Entwurf verteilter Netz"-werk"-an"-wen"-dung"-en.
Es setzt auf \acs{HTTP} und \acs{HTTPS}-Methoden und stellt verschiedene Methoden zur Verfügung.
Für den Einsatz von \acs{REST}-Services kommen folgende \acs{CRUD}-Prinzipien in Frage:

\begin{table}[ht]
	\renewcommand{\arraystretch}{1.2}
	\centering
	\begin{tabular}{l|l|l}
		\acs{CRUD} & \acs{HTTP} & Beispiel-\acs{URI} \\
		\hline
		Create & POST & http://lolcalhost:8080/cinema-system/movie \\
		Read & GET & http://lolcalhost:8080/cinema-system/show/1 \\
		Update & PUT & http://lolcalhost:8080/cinema-system/movie/5 \\
		Delete & DELETE & http://lolcalhost:8080/cinema-system/customer/4711 \\
	\end{tabular}
	\caption{\acs{CRUD}-Befehle und deren Verwendung}
	\label{tab:crud}
\end{table}

Wir die \acs{REST}-Schnittstelle mit einem \textit{POST} aufgerufen wird ein neuer Datensatz in der Datenbank angelegt.
In dem Beispiel aus Tabelle \vref{tab:crud} wird ein neuer Film in der Datenbank angelegt.
Die benötigten Informationen des Films werden über einen Form-Parameter (\textit{@FormParam}) im \acs{JSON}-Format an die \acs{REST}-Schnittstelle übertragen. %TODO Beispiel bringen
Wird in der Ressource die Schnittstelle mit einem \textit{GET} aufgerufen, wird ein Datensatz angefragt, der über einen anderen Webservice die Anforderung an die Datenbank sendet und anschließend wieder an die Ressource zurücksendet.
In dem Beispiel aus Tabelle \vref{tab:crud} wird die Ressource Film aufgerufen und der Datenbankeintrag mit der Film-ID 1 abgefragt.
Mittels \textit{PUT} wird ein vorhandener Datensatz aktualisiert.
In dem Beispiel aus Tabelle \vref{tab:crud} würde z.B. ein neuer Schauspieler hinzugefügt werden.
Schlussendlich löscht man einen Datensatz, indem man in der Ressource die Schnittelle mit \textit{DELETE} aufruft.
In dem Beispiel aus Tabelle \vref{tab:crud} würde man den Kunden mit der Kundennummer 4711 aus der Datenbank entfernen.\\

Während Anfragen über die \acs{REST}-Schnittstelle über den Server beantwortet werden, können auch verschiedene \acs{HTTP} bzw. \acs{HTTPS}-Status zurückgeliefert werden (siehe Tabelle \vref{tab:http_status}).

\begin{table}[!hpt]
	\renewcommand{\arraystretch}{1.2}
	\centering
	\begin{tabular}{l|p{8.5cm}}
		Status Code & Beschreibung \\
		\hline
		200 (OK) & Alles OK bei der Verbindung \\
		400 (Bad Request) & Der Server weiß nicht, was er mit der Anfrage machen soll \\
		401 (Unauthorized) & Der Client muss sich erst authentifizieren, z.B. falsches Passwort gewählt \\
		501 (Media Type unsupported) & Der gewählte Medientyp wird nicht unterstützt z.B. Text Plain oder es wird ein anderes Format erwartet z.B. \acs{JSON} \\
		503 (Service Unavailable) & Anfrage konnte nicht abgeschlossen werden
	\end{tabular}
	\caption{Mögliche \acs{HTTP}-Status eines Webservices}
\label{tab:http_status}
\end{table}
\clearpage

\subsection{Annotation für eine \acs{REST}-Schnittstelle}
\label{ssec:annotationen_schnittstelle}
\authorsection{\authorSG}
Das Paket JAX-RS von Jersey bietet eine Vielzahl an Möglichkeiten für Annotationen einer \acs{REST}-Schnittstelle. In diesem Kapitel sollen einige im Projekt verwendete vorgestellt werden.

\subsubsection*{@Path}
Mittels \textit{@Path} werden die jeweiligen Pfade zu den \acs{REST}-Schnittstellen in einer Ressource dargestellt. Diese können nach belieben geschachtelt werden. \\
So ist es z.B. möglich auf dem Server den Pfad zur Ressource der Reservierung anzugeben \textit{@Path("reservation")}. 
Hier werden dann noch weitere Pfade wie \textit{@Path("book")} angegeben, der ausschließlich für das Buchen von Tickets oder \textit{@Path("block")}, der ausschließlich für das Blocken von Sitzen verantwortlich ist. In ihnen sind dann verschiedenen \acs{CRUD}-Befehle definiert (vgl.~\nameref{sss:get}, \nameref{sss:post}, \nameref{sss:delete} auf Seite~\pageref{sss:delete}) 

\subsubsection*{@MediaType}
\label{sss:mediatype}
Die in diesem Projekt verwendete Annotation \textit{@MediaType} deklariert in jeder Ressource, welche Austauschformate zulässig sind. \\
Deklariert sind in diesem Projekt: \textit{MediaType.TEXT\_PLAIN\_TYPE}, sodass über das Front-End eine ID mitgegeben werden kann; \textit{MediaType.APPLICATION\_JSON\_TYPE}, sodass sichergestellt ist, dass nur ein \acs{JSON} als Austauschformat zulässig ist.

\subsubsection*{@Consumes}
\label{sss:consumes}
Mit Hilfe der Annotation \textit{@Consumes} definiert man an der \acs{REST}-Schnittstelle, dass eine Verarbeitung nur mit dem gewünschten MediaType erfolgt. Wird dies nicht erfüllt, kommt die Fehlermeldung 501 (vgl. Tabelle \vref{tab:http_status}). 

\subsubsection*{@FormParam}
\label{sss:formparam}
Über die Annotation \textit{@FormParam} wird in der \acs{REST}-Schnittstelle festgelegt, dass das \acs{JSON} aus dem Front-End ein bestimmter String vor dem eigentlichen \acs{JSON} erwartet wird. \\
Ein korrektes \acs{JSON} mit \textit{@FormParam} für das Blocken würde z.B. folgendermaßen aussehen: \textit{block=\{ <DATEN> \}}
\subsubsection*{@GET}
\label{sss:get}
Über die \textit{@GET}-Annotation wird eine einfache Anfrage erstellt. So ist sichergestellt, dass auch nur die \acs{REST}-Schnittstelle aufgerufen wird, die die \textit{@GET}-Annotation hat, obwohl es vielleicht nochmals die selbe Methode gibt für einen POST. 

\subsubsection*{@POST}
\label{sss:post}
Über die \textit{@POST}-Annotation wird sichergestellt, dass wirklich das übergebene \acs{JSON} in die in diesem Projekt verwendete Datenbank gespeichert werden soll. 

\subsubsection*{@DELETE}
\label{sss:delete}
Über die \textit{@DELETE}-Annotation wird sichergestellt, dass wirklich der übergebene \textit{String} z.B. eine ID in der in diesem Projekt verwendeten Datenbank gelöscht werden soll. 

\subsubsection*{@Produces}
\label{sss:produces}
Über die Annotation  \textit{@Produces} wird an der \acs{REST}-Schnittstelle definiert, dass ausschließlich die deklarierten Medien-Typen wie z.B. TEXT\_PLAIN oder JSON erzeugt werden (vgl. \nameref{sss:mediatype}).