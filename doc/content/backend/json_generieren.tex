\section{Generieren der \acf{JSON}}
\label{sec:json_generieren}
\authorsection{\authorSG}

Java ist eine objektorientierte Sprache.
Alle Klassen sind dementsprechend als Objekte hinterlegt.
Durch die Implementierung eines sog. Object-Mappers wie dem Jackson-Mapper\footnote{\url{https://github.com/FasterXML/jackson}}
ist es möglich die zuvor erstellten Objekte in eine \acs{JSON}-Datei zu überführen.
Er erkennt automatisch, ob es sich um ein Objekt, ein Attribut oder eine Liste handelt.
Der Entwickler ruft lediglich die Methode \jinline |writeValueAsString(Object-to-JSON)|, die ein Objekt als Übergabeparameter erwartet auf. \\
Das Auslesen eines \acs{JSON}-Strings wird mittels der Methode \jinline |readValue(json, javaClass)| ausgeführt. 
Diese erwartet als Übergabeparameter ein \acs{JSON}, sowie die Java-Klasse des Objekts, in es gewandelt werden soll z.B. \jinline |Show.class|. 
Der Vorteil dieser Methode ist, dass hier die Java-Klasse angegeben wird, also das Objekt, in das der \acs{JSON}-String konvertiert werden soll.

Der Jackson-Mapper wird in diesem Projekt über eine Dependency zu einem Maven-Projekt hinzugefügt und kann anschließend verwendet werden (siehe Quelltext \ref{lst:Einbindung_ObjektMapper}) %TODO varioref

\begin{minipage}{\linewidth}
\begin{lstlisting}[language=XML]
<dependency>
	<groupId>com.fasterxml.jackson.core</groupId>
	<artifactId>jackson-databind</artifactId>
	<version>2.9.4</version>
</dependency>
\end{lstlisting}
\captionof{lstlisting}{Einbindung des Objekt-Mappers in die pom.xml}
\label{lst:Einbindung_ObjektMapper}
\end{minipage}

\begin{minipage}{\linewidth}
	\begin{lstlisting}[style=lstJava]
	public static String toJSON ( Object object ) throws JsonProcessingException
	{
		String str = "";
		ObjectMapper om = new ObjectMapper();
		om.getSerializerProvider().setNullKeySerializer(nullKeySerializer);
		str = om.writeValueAsString(object);
		return str;
	}
	
	public static Object fromJSON ( String json, Class<?> javaClass ) throws IOException
	{
		ObjectMapper om = new ObjectMapper();
		om.configure(DeserializationFeature.FAIL_ON_UNKNOWN_PROPERTIES, false);
		Object obj = om.readValue(json, javaClass);
		return obj;
	}
	\end{lstlisting}
	\captionof{lstlisting}{Ausschnitt aus der selbst erstellten Klasse JSONConverter}
\end{minipage}