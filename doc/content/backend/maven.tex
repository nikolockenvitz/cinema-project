\section{Maven}
\label{sec:maven}
Für die Entwicklung des Kinoreservierungsprogramms wird ein Maven-Projekt verwendet. Maven sorgt als übergeordnetes Projekt dafür, dass in allen Java-Projekten die gleichen JAR-Dateien sind. Sprich die Dateien, die benötigt werden um die Java-Klassenbibliotheken zu nutzen. Die benötigten Dateien werden mittels sog. Dependencies in einem Mavenprojekt eingebunden (siehe Quelltext \vref{lst:Einbindung_ObjektMapper}). Jedes Maven-Projekt enthält eine pom.xml-Datei, die die Informationen und Konfigurationen dieses Projekts enthält. 

Ein weiterer Vorteil der Verwendung eines Maven-Projekts liegt darin, dass im Falle einer neuen Version der verwendeten Dependency\footnote{notwendige Zusatzbibliotheken wie z.B. Datenbanktreiber} lediglich die Versionsnummer geändert werden muss und Maven automatisch die aktuelle Version aus dem Internet lädt und den jeweiligen Projekten zur Verfügung stellt. Somit muss der Programmierer nicht manuell die Dateien herunterladen und dem Build-Path im Java hinzufügen. 

Eine weitere Möglichkeit besteht darin, dass man in der pom.xml definieren kann, welches Format am Ende des Kompilierens erwünscht ist. Hier kann er z.B.  zwischen einer pom.xml, welche die Daten in den anderen Java-Projekten bereitstellt und einer WAR-Datei wählen. Letztere wird verwendet, um die Applikation auf einem in diesem Projekt verwendeten Tomcat-Server laufen zu lassen. 

Als Struktur für das Backend dieses Projekts wurde ein Parent-, eine Shared-, eine System- und eine Data-Maven-Projekt angelegt. Diese fungieren gleichzeitig auch als Java-Klassen. In dem Parent-Projekt wird u.a. deklariert, welche Depenencies für das Projekt genutzt werden soll und mit welcher Version Maven verwaltet werden soll. Die Shared-Klasse dient dazu die von der Parent-Klasse zur Verfügung gestellten JAR-Dateien und gemeinsam genutzten Java-Klassen in den Haupt-Klassen Data und System bereitzustellen.

In den Haupt-Klassen System und Data wird über die zuvor getätigte Konfiguration \emph{<packaging>war</packaging>} in der pom.xml jeweils eine WAR-Datei erzeugt. Diese werden anschließend auf den Tomcat-Server kopiert und dort vollautomatisch entpackt. Sie beinhaltet sämtlichen Code der jeweiligen Klasse, der Notwendig ist, um damit Zugriff auf die zuvor definierten REST-Schnittstelle (Kapitel \vref{sec:rest}) zu gewährleisten. 