% !TEX root =  master.tex
\section{Authentifzierung}
\authorsection{\authorNL}
Kunden sollen die Möglichkeit haben, sich zu registrieren.
Dabei setzen sie ein Passwort, welches im Anschluss gemeinsam mit der E-Mail-Adresse zur sicheren Authentifizierung dient.

Das Passwort wird dabei nicht im Klartext gespeichert, sondern in Form eines Hash\-wertes. % TODO: Details Hash + Quelle
Bei der Berechnung des Hash\-wertes wird ein benutzer\-spezifischer Salt mit dem Passwort verknüpft, um die Passwörter gegen Angriffe mit \enquote{Rainbow Tables} abzusichern. % TODO: Erläuterung + Quelle
Der Hash aus Salt und Passwort wird in der Datenbank gespeichert, sodass ein Ausschnitt der Datenbank in vereinfachter Form wie folgt aussehen könnte:

\setlength{\tabcolsep}{6pt}
\begin{tabular}{|l|l|l|}
	\hline
	email & salt & hashSHA1 \\
	\hline
	max@muster.de & 1952e9 & 3b9c4352194e71abfc3931f6634ef6355018149b \\
	\hline
	erika@musterfrau.eu & d4aeff & f8b2f89758a31e17228233b5bd26b7267aa92ea8 \\
	\hline
\end{tabular}

In beiden Fällen ist das Klartextpasswort \enquote{geheim}.
Durch den Salt ist dies aber nicht direkt erkennbar.

Wenn sich nun ein Benutzer versucht anzumelden, gibt er das Passwort an.
In der Verarbeitungsschicht muss dann zum betreffenden Nutzer der Salt herausgesucht werden, mit dem eingegebenen Passwort verknüpft und anschließend gehasht werden.
Stimmt dieser Hash mit dem aus der Datenbank überein, so war das Passwort korrekt und der Nutzer konnte seine Identität erfolgreich bestätigen.

Wenn sich ein Kunde mit seinem Benutzerkonto erfolgreich anmeldet, wird eine \textit{Session\-ID} erzeugt.
Diese wird sowohl in der Datenbank und als auch in Form eines Cookies im Webbrowser des Benutzers gespeichert, damit im folgenden Verlauf einerseits die Authentifizierung gewährleistet bleibt und andererseits das Passwort nicht mehr benutzt werden muss. % TODO: Quelle

Jeder benutzer\-spezifische Aufruf der \acs{API} muss dann eine eindeutige Identifikation des Benutzers, z.B. die E-Mail-Adresse und die \textit{SessionID} des Benutzers enthalten, damit in der Verarbeitungsschicht sichergestellt werden kann, dass die auszuführende Aktion auch autorisiert ist. % TODO: Erläuterung am Beispiel
