% !TEX root =  master.tex
\section{Sitzplatzauswahl und Blockierungen}
\authorsection{\authorNL}
Beim Reservierungsvorgang ist es essentiell, dass ein Sitzplatz nicht mehrfach reserviert wird.

Hier ein einfaches Beispiel:
Benutzer A besucht die Seite, wählt einen Film und eine Vorstellung aus.
Die Saalübersicht wird geladen, wobei Sitzplätze, die bereits reserviert sind, optisch hervorgehoben sind und nicht mehr ausgewählt werden können.
Etwa zur gleichen Zeit besucht Benutzer B die Seite.
Er wählt die selbe Vorstellung, sieht die gleiche Saalübersicht und wählt einen Platz aus: Reihe 4, Platz 2.
Benutzer B folgt daraufhin den weiteren Schritten, bezahlt seinen Sitzplatz und erhält eine Bestätigung, dass Reihe 4, Platz 2 erfolgreich von ihm gebucht wurde.
Auch Benutzer A hat Reihe 4, Platz 2 ausgewählt.
Zu dem Zeitpunkt als Benutzer A den Saalplan geladen hat, war die Reservierung von Benutzer B noch nicht abgeschlossen, sodass der Platz frei war.
Auch Benutzer A möchte nun das Ticket bezahlen, muss aber eine Fehlermeldung erhalten, damit das Ticket für Reihe 4, Platz 2 nicht doppelt verkauft wird.

Spätestens bei der verbindlichen Reservierung und der normalerweise folgenden Bestätigung muss eine Fehlermeldung erscheinen, wenn der Platz bereits belegt ist.
Im Idealfall wird der Nutzer aber schon früher benachrichtigt, wenn er einen Platz ausgewählt hat, der in der Zwischenzeit von einem anderen Benutzer reserviert wurde.

Der Fall, dass ein Platz durch einen anderen Benutzer reserviert wird, ist mit gewisser Wahrscheinlichkeit vorhersehbar.
Wählt ein Benutzer im Saalplan ein paar Plätze aus, so wird er vermutlich in den nächsten paar Minuten mit der Bezahlung fortfahren und die Plätze für sich reservieren.
In dem Moment, wo ein Benutzer einen Sitzplatz also nur auswählt, kann diese Information bereits in der Datenbank gespeichert werden.
Der Sitzplatz wird für diesen Benutzer zwar nicht reserviert, aber für einen kurzen Zeitraum vorgemerkt.
Will nun jemand einen Platz auf dem Saalplan auswählen, teilt er dies über die \acs{API} dem Back-End mit.
Nun gibt es mehrere mögliche Ergebnisse:
\begin{enumerate}
\item Der ausgewählte Sitzplatz ist bereits reserviert.
\item Der ausgewählte Sitzplatz ist bereits für einen anderen Benutzer vorgemerkt, könnte also in einigen Minuten wieder frei werden.
\item Der Sitzplatz ist weder reserviert, noch vorgemerkt.
Er wurde nun erfolgreich für den aktuellen Benutzer vorgemerkt.
\end{enumerate}

Dies kann nun dazu führen, dass ein Benutzer einen freien Saalplan angezeigt bekommt und beim Anklicken eines Platzes dieser nun nicht mehr als \enquote{frei}, sondern als \enquote{durch einen anderen Nutzer belegt} angezeigt wird.

Die zu der Reservierung zugehörigen Operationen auf der Datenbank müssen als Transaktion ausgeführt werden, sodass die Reservierung entweder erfolgreich war oder eine Fehlermeldung den Benutzer informiert, dass die Reservierung fehlgeschlagen ist.

Alle Sicherheitsprüfungen müssen im Back-End ausgeführt werden.
Natürlich können auch einzelne Prüfungen im Front-End implementiert werden, diese helfen allerdings nur einem normalen Benutzer den Vorgang zu erleichtern.
Eine Absicherung bieten derartige Prüfung im Front-End nicht, da der clientseitig ausgeführte Quellcode logischerweise vom Client verändert werden kann und jeder Angreifer eine manipulierte Anfrage mit selbst gewählten Parametern an die \acs{API} schicken kann.