% !TEX root =  master.tex
\section{Kritische Reflexion und Bewertung}
\multipleauthorsection{\authorRF}{\authorEJ}

\subsection{Funktionalität und Qualität der Software}
\authorsection{\authorRF}

Das Front-End wurde erfolgreich nach den eigenen Wünschen mit Hilfe der zuvor festgelegten Iterationsschritten umgesetzt.
Jedoch ist das aktuelle Design an diversen Stellen inkonsistent und weist sowohl farbliche als auch formtechnische Ungleichheiten auf. \\
Dies ist vor allem an der Sitzplatzauswahl zu erkennen, da die Darstellung nicht mit Design der sonstigen Website übereinstimmt.
<<<<<<< HEAD
Jedoch schränkt es das Endprodukt sowie die Zielvorgabe nicht ein und ist höchstens als ein temporärer Makel.
=======
Zudem können allgemein nur Bilder mit festgelegten Dimensionen im Karoussell eingesetzt werden, da sich sonst beim Wechsel der Inhalt verschiebt und das Nutzerlebnis beeinträchtigt.
Die Bilder der Filme allgemein können bisher nur in einem festgelegten Format (JPG) gespeichert werden, da sie sonst nicht referenziert werden und nicht dargestellt werden auf der Seite.
Jedoch spiegelt der Entwicklungsstand nicht das Endprodukt wieder und ist nur ein temporärer Makel.
>>>>>>> branch 'master' of https://MatrixS@bitbucket.org/fallstudie/cinema-project.git

Das Back-End wurde ebenso mit Hilfe der Iterationsschritte erfolgreich nach den eigenen Wünschen umgesetzt.
Jedoch weist es einige Schwächen auf.
Die Schwächen beziehen sich vorwiegend auf die Komplexität des Back-Ends im Bezug auf die Größe des gesamten Projekts.
So wurde das Drei-Schichten-Modell implementiert, was die Schwierigkeit beim Entwickeln und Benutzen erhöht hat.\\
%Das in den Anforderungen als optional beschriebene Nutzen einer Datenbank hat somit einen großen Teil der Entwicklungszeit gekostet, welcher nicht für andere ebenfalls wichtige Entwicklungen oder Planungen genutzt werden konnte.

\subsection{Zusammenarbeit im Team}
\multipleauthorsection{\authorRF}{\authorEJ}

Die Gruppenarbeit ist allgemein positiv zu werten. 
So konnten alle Beteiligten Einblicke in möglicherweise unbekannte Bereiche gewinnen oder sich weiterbilden und den Kenntnisstand weiter ausbauen.
Des Weiteren war die Motivation bei der Seminararbeit sowie der Implementierung erhöht, da mit einem erfolgreichen und guten Projekt die Benotung derselbigen gut ausfallen sollte.

Dies führte auch dazu, dass ein großer Anteil der Gruppenarbeit auf individueller Basis entstanden ist und die Projektarbeit auf diesen aktuellen Stand gehoben hat. Also ein voll funktionales Back-End mit einer Datenbank zur Datenhaltung inklusiver der Verwendung von Webservices sowie ein Front-End mit diversen nicht vorgegebenen Features. Diese beiden Bereiche können als Alleinstellungsmerkmal angesehen werden.\\
Jedoch wurde dadurch die Zusammenarbeit oftmals in den Hintergrund gerückt und es wurden Entscheidungen, welche vom Team demokratisch gefällt werden sollten, durch wenige Personen entschieden und eine Umsetzung angefangen. Dennoch wurden auch einige Prozesse in Teams erarbeitet, was oftmals das Verstehen des Sachverhalts erleichtert hat.
Durch das Übergehen verschiedener Personen bei Entscheidungen kam es auch schnell, u.a. auch durch fehlende Ambitionen, zu Demotivierungen, welche wiederum einseitige Arbeitsverhältnisse zur Folge hatten.

Trotzdem war die Arbeitsatmosphäre in den entstanden Gruppen und darüber hinaus recht entspannt, da im Team beschlossene Aufgaben in einem eigenen Arbeitstempo erledigt werden konnten und auch auf Schwächen einzelner Individuen Rücksicht genommen werden konnte.
% sollte man vielleicht nicht so drastisch formulieren -> "Gruppen-Abspaltung" etc. ist eher negativ konnotiert
Ferner wurden durch eine unsinnvolle Aufspaltung der Arbeit die gegebenen Ressourcen nicht erfolgreich genutzt und hätten das Projekt weiter fördern können.

Positiv hervorzuheben ist andererseits der erste Sprint, da hier erfolgreich als Team auf ein Ziel hin gearbeitet wurde und jeder seine individuellen Stärken einsetzen konnte.
Dies hatte einen schnellen Start zur Folge, der die vorher beschriebenen Problem erst ermöglicht hat.
Wobei hier eine klare Richtung des Projekts zu erkennen war, was allerdings durch ein fehlendes Projekt-Management im Team dazu geführt hat, dass die Energie und die Ambitionen nicht aufgegriffen wurden und in ein noch besseres und erfolgreicheres Projekt umgewandelt werden konnten.

Endgültig ist zu sagen, dass die Autoren als Team aus den Erfahrungen gelernt haben und nun in weiteren Projekten ein sinnvolles Projektmanagement umsetzen werden.
Dadurch sollte nicht nur die Motivation aller Team-Mitglieder gesteigert, sondern auch die Produktivität erhöht werden.
Diese Probleme und Lösungen sind das Ergebnis eines konstruktiven Feedbackgesprächs, welches am Ende des Projekts geführt wurde und bei dem jedes Mitglied seine Erfahrungen und Erkenntnisse teilen konnte.
