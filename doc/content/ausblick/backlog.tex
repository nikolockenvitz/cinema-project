% !TEX root =  master.tex
\section{Backlog und Theoretischer Sprint 3}

\subsection{Theoretische Sprint 3}
Die größte Neuerung des angedachten dritten Sprintes ist die komplette Anbindung des Mitarbeiters. 

Der Mitarbeiter besteht aus Vorname, Nachname, E-Mail-Adresse, Mitarbeiternummer und einem dem aktuellen Sicherheitsstandards genügenden Passwort.
Grundlagen dazu wurden bereits im vorhergehenden Sprint gelegt, eine Mitarbeitertabelle ist in bereits der Datenbank vorhanden.
Teile der Businesslogik sind ebenfalls bereits vorhanden, namentlich das \acs{DTO} und die Entität mit der zugehörigen Konvertierung.
Diese wurde ebenfalls bereits mithilfe von Tests überprüft.

Eine Neuerung wäre eine spezielle Ansicht des Front-Ends angepasst an die Bedürfnisse der Mitarbeiter. 
Hierzu zählen eine einfache Übersicht der aktuell laufenden Filme, eine simple und schnelle Suche nach Reservierungen und ein Kalendersteuerelement der nächsten vierzehn Spieltage ihm Kino.
Die Anordnung der Filme auf der Startseite des Mitarbeiters sollte nach Beliebtheit und aktueller Uhrzeit erfolgen, sodass Primetime-Filme eine bestimmte Zeit vor Start des Blockbusters, aber auch zur jeweiligen Tageszeit eine passende Filmauswahl angezeigt werden.

Dies sollte den Buchungsvorgang vor Ort erheblich erleichtern und Teilzeitkräften sowie Ferienarbeiter den Einsteig in das System auch ohne lang Einarbeitungszeit ermöglichen.
Sobald ein Mitarbeiter eine Vorstellung ausgewählt hat, bekommen er und der Kunde vor Ort einen aktuellen Saalplan angezeigt.
Nachdem der Kunde nun seine gewünschten Sitze ausgewählt hat, beginnt der Bezahlvorgang, dieser erfolgt über ein bereits vorhandenes Kassensystem. Beendet wird die Buchung schlussendlich durch das Ausdrucken der Kinotickets.

Aufgrund der Bezahlung vor Ort muss die Mitarbeiteransicht in zweifacher Sicht geschützt werden.
Einerseits sollte das System nur innerhalb des lokalen Netzwerks verfügbar sein.
Andererseits ist eine Sicherheitsprüfung, in Form einer Authentifizierung, relevant, sodass kein unbefugter Zutritt auf die internen Bezahlsysteme erlaubt wird.

Zusätzlich zum Hauptaspekt des dritten Sprints gehören kleinere Verbesserungen sowie Fehlerbeseitigungen. 
Diese bestehen aus einer verbesserten Bestätigungsseite der Online-Buchung in welcher neben der Klassifikation der Sitze, auch eine Nummerierung dargestellt und eine generierte PDF-Datei mit Informationen der Bestellung zum Download angeboten werden sollte.

\subsection{Backlog}
Für die langfristige Planung des Projektes wurden bereits Backlog-Items angelegt, die in späteren Sprints aufgefasst werden soll.
Eine hohe Priorität hält dabei die Überprüfung und langfristige Speicherung der Gutscheincodes mithilfe der Datenbank inne.
Dabei sollen verschiedene Arten von Gutscheinen, sowie deren Ablaufdatum und vergangene Transaktionen gespeichert werden. 

Die Darstellung der Sitze soll ebenfalls in einem zukünftigen Sprint angepasst werden, sodass einerseits Breite und Höhe eines Sitzes für die Darstellung mit gespeichert werden, andererseits aber auch Sitzflächen für Rollstuhlfahrer o.Ä. ausgewiesen werden.

Die Kommunikation zwischen Front- und Back-End muss erweitert werden, dabei sollten regelmäßige Abfragen das temporäre Speichern verschiedener Übergabeparameter ersetzen.

Des Weiteren sollte ein breiteres Portfolio an Zahlungsmöglichkeiten das Benutzererlebnis verbessern und dem Kunden erlauben, ohne vorherigen Besuch des Kinos, online zu bezahlen.

Als Alternative zu den weiteren Zahlungsmöglichkeiten, sollte die Option einer Reservierung gegeben sein.
Wobei das System der Vergabe einer Reservierungsnummer sicherer gestaltet werden sollte, momentan werden diese noch linear vergeben, aber sollten durch eine scheinbar chaotische Weise ersetzt werden.
Um Kunden eine unverbindlichere und bequemere Art der Vorbestellung zu ermöglichen.

Einhergehend zur Reservierung, sollte eine Stornierung dieser, als auch eine Buchung möglich sein.

Um Reservierungen und Buchungen für einen passionierten Kinogänger einsehbar und verwaltbar (u.a. Stornierung) zu gestalten, wird ein Login eingerichtet. 
Über diesen kann eine Personalisierung der Darstellung sowie Filmempfehlungen die User Experience verbessern.


