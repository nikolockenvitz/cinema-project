% !TEX root =  master.tex
\section{Fazit zur Gruppen- und Seminararbeit}
\multipleauthorsection{\authorRF}{\authorEJ}

Durch die Gruppenarbeit konnte das Team viele positive Effekte einer Zusammenarbeit mitbekommen, zu diesen zählt einerseits die Leistungsstärke einer Gruppe.
Allein durch ein einzelnes Mitglied des Teams hätte niemals ein so fortgeschrittenes Modell eines Kinobuchungssystems erreicht werden können.
Andererseits konnte durch das Zusammenführen von verschiedenen Vorkenntnissen und Kompetenzen jedes Gruppenmitglied einen positiven Lerneffekt erlangen und dieses Wissen in die gestellte Aufgabe einbringen um ein Ergebnis zu erlangen, worauf jeder stolz sowie zufrieden sein kann.
Des Weiteren führt gegenseitiger Respekt zu einer positiven Atmosphäre, dabei stellt sich ein motivierendes und bildendes Lernklima ein.
Aufgrund wessen Entscheidungen innerhalb der Gruppe einfacher akzeptiert und mitgetragen wurden konnten.
Dieses Phänomen der Gruppendynamik ist nicht nur eine treibende Kraft für die gemeinsame Aufgabe, sondern bietet jeden einzelnen Gruppenmitglied Sicherheit und Zuversicht in sich selbst sowie andere.

Da jeder positive Effekt meistens auch noch einen negativen Beigeschmack hat auch diese Gruppenarbeit die möglichen Nachteile einer Zusammenarbeit aufgedeckt.
Einzelne Personen können eine sehr dominante Rolle innerhalb der Gruppe einnehmen und diese führen dabei zu einer Unterdrückung von Meinungen und erwürgen diese in keinem. 
Aus diesem Grund kann es schnell zu einem Gruppendenken kommen, die einzelnen Mitglieder denken nicht mehr selbst nach und folgen der Meinung anderer, obwohl es für das Problem wohl möglich noch bessere Lösungen gibt, die sonst noch keiner erwähnt hat.
Letztendlich ist die wohl komplexeste Aufgabe innerhalb eines Teams keine Spannungen entstehen zu lassen und falls sie doch entstehen sollten, diese bereitwillig zu beseitigen und sich vor keiner Konfrontation zu scheuen.
Dabei kann es schnell zu einer Minderung der Arbeitsqualität und Motivation kommen. 

Um einen bestmöglichen Effekt aus der Gruppenarbeit zu ziehen ist die Motivation einzelner Gruppenmitglieder, eine Koordination der Arbeitsressourcen, verschiedene Diskussionen für die Entscheidungsfindung sowohl das Vermeiden von Gruppendenken wichtig und notwendig.


Im Allgemeinen ist die Gruppenarbeit durch ein positives Arbeitsklima gekennzeichnet gewesen.
Allein durch die gesammelten Erfahrungen und Erkenntnisse konnten Probleme erkannt und Lösungen dazu erarbeitet werden.
Diese Probleme frühzeitig zu Erkennen und zu beheben ist nun für die Autoren der Arbeit einer der wichtigsten Aspekte einer Gruppenarbeit geworden.
Somit wurde durch ein Kinoreservierungssystem nicht nur der Gegenwert der Planung in einem Projekt, sondern auch die Wichtigkeit des Team-Managements und die Bedeutsamkeit der Kommunikation in Projekten mit mehreren Personen verdeutlicht.

