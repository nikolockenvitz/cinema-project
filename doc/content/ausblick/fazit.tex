% !TEX root =  master.tex
\section{Fazit zur Gruppen- und Seminararbeit}
\multipleauthorsection{\authorRF}{\authorEJ}

Durch die Gruppenarbeit konnte das Team viele Erfahrungen in der Zusammenarbeit sammeln, zu diesen zählt unter anderem die Leistungsstärke einer Gruppe.
Allein durch ein einzelnes Mitglied des Teams wäre niemals ein so fortgeschrittenes Modell eines Kinobuchungssystems erreicht worden.
%Da jeder positive Effekt meistens auch mit einem negativen Beigeschmack einher geht, hat auch diese Gruppenarbeit die möglichen Nachteile einer Zusammenarbeit aufgedeckt.
%So wurde zu verschiedenen Zeitpunkten die Arbeitsmoral durch fehlende Ambitionen gesenkt.
Durch das Zusammenführen von verschiedenen Vorkenntnissen und Kompetenzen eines jeden Gruppenmitglieds ein positiver Lerneffekt erlangt und vorhandenes Wissen in die gestellte Aufgabe eingebracht werden.
%Somit wurde ein Ergebnis erreicht, über das jeder sowohl stolz als auch glücklich sein kann.
Allerdings wurden auch kleinere Entscheidungen von Teilen der Gruppe ohne Absprache im gesamten Team beschlossen und führten so zur zeitweisen Demotivation einzelner Mitglieder.
Nichtsdestotrotz führte gegenseitiger Respekt zu einer positiven Atmosphäre, dabei stellte sich ein motivierendes und bildendes Lernklima ein.
%Aufgrund dessen wurden Entscheidungen innerhalb der Gruppe einfacher akzeptiert und mitgetragen.
Dieses Phänomen der Gruppendynamik ist nicht nur eine treibende Kraft für die gemeinsame Aufgabe, sondern bietet jedem Gruppenmitglied Sicherheit und Zuversicht in die eigene Arbeit sowie die Arbeit der Anderen.

%Letztendlich war die wohl komplexeste Aufgabe innerhalb eines Teams keine Spannungen entstehen zu lassen bzw. diese so schnell wie möglich zu beseitigen.
%Trotzdessen ist die Gruppenarbeit durch ein positives Arbeitsklima gekennzeichnet gewesen.
Allein durch die gesammelten Erfahrungen und Erkenntnisse konnten Probleme erkannt und Lösungen dazu erarbeitet werden.
Diese Probleme frühzeitig zu Erkennen und zu beheben ist nun für die Autoren der Arbeit einer der wichtigsten Aspekte einer Gruppenarbeit geworden.
Somit wurde durch ein Kinoreservierungssystem nicht nur der Gegenwert der Planung in einem Projekt, sondern auch die Wichtigkeit des Team-Managements und die Bedeutsamkeit der Kommunikation in Projekten mit mehreren Personen verdeutlicht.
