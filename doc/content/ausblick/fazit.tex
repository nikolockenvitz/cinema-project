% !TEX root =  master.tex
\section{Fazit zur Gruppen- und Seminararbeit}
\multipleauthorsection{\authorRF}{\authorEJ}

Durch die Gruppenarbeit konnte das Team viele Erfahrungen in der Zusammenarbeit sammeln, zu diesen zählt unter anderem die Leistungsstärke einer Gruppe.
Allein durch ein einzelnes Mitglied des Teams wäre niemals ein so fortgeschrittenes Modell eines Kinobuchungssystems erreicht worden.

Durch das Zusammenführen von verschiedenen Vorkenntnissen und Kompetenzen eines jeden Gruppenmitglieds konnte ein positiver Lerneffekt erlangt und vorhandenes Wissen in die gestellte Aufgabe eingebracht werden.
Somit wurde ein Ergebnis erreicht, über das jeder stolz sein kann.
Allerdings wurden auch kleinere Entscheidungen von Teilen der Gruppe ohne Absprache im gesamten Team beschlossen und hatte eine zeitweise Demotivation einzelner Mitglieder zur Folge.
Nichtsdestotrotz führte gegenseitiger Respekt zu einer positiven Atmosphäre, dabei stellte sich ein motivierendes und bildendes Lernklima ein.
Dieses Phänomen der Gruppendynamik ist nicht nur eine treibende Kraft für die gemeinsame Aufgabe, sondern bietet jedem Gruppenmitglied Sicherheit und Zuversicht in die eigene Arbeit sowie die Arbeit der Anderen.

Allein durch die gesammelten Erfahrungen und Erkenntnisse konnten Probleme erkannt und Lösungen dazu erarbeitet werden.
Hierbei war die wohl komplexeste Aufgabe innerhalb des Teams keine Spannungen entstehen zu lassen bzw. diese so schnell wie möglich zu beseitigen.
Letztendlich wurde durch ein Kinoreservierungssystem nicht nur die Relevanz der Planung in einem Projekt, sondern auch die Wichtigkeit des Team-Managements und die Bedeutsamkeit der Kommunikation in Projekten mit mehreren Personen verdeutlicht.
