% !TEX root =  master.tex
\section{Testverfahren}
Testen wird oftmals als Prozess, der aufzeigen soll, dass keine Fehler in dem Programmcode vorhanden sind, fehl verstanden. Im eigentlichen Sinne geht es dabei nicht darum zu zeigen, dass der Quellcode fehlerfrei ist, sondern das er Fehler enthält und nach diesen gesucht werden muss.\footnote{\url{http://www.knaupes.net/theorie-der-softwaretests/}}

\subsection{Testprinzipien}
Aufgrund dieser Definition für Testen müssen Ausgangsvoraussetzungen gegeben sein um eine möglichst umfangreiche Testabdeckung gewährleisten zu können. Zu diesen zählen das Feststellen, ob der Quellcode genau die Anforderungen erfüllt und sonst auch keine weiteren Eventualitäten abdeckt. 
Ebenfalls sollte dabei genau dokumentiert werden, welcher Testfall bereits abgedeckt wurde. 
Da bei vielen unterschiedlichen Testfällen und einem größeren Testpensum der Überblick schnell verloren werden kann.
Des Weiteren müssen Testfälle reproduzierbar sein und es darf sich nicht um einzelnes Phänomen handeln. 
Einen Testfall zu generieren, der aber eigentlich gar nichts mit dem zu testenden Quellcode zu tun hat ist einerseits nicht möglich und andererseits nicht sinnvoll. 
Der aber wohl wichtigste Punkt, ob fachlich oder menschlich gesehen ist jedoch, dass der Tester nicht der Programmierer selbst sein darf.
Als Programmierer des Quellcodes besitzt man eine voreingenommene Meinung, der Quellcode hat genau einen Zweck und diesen erfüllt er für den Programmierer auch. 
Daher kann ein größerer Blickwinkel auf den Quellcode leider nicht gewährleistet werden und die Testfälle können nicht in hinreichender Genauigkeit generiert werden. 
Ebenfalls hat ein Tester auch keine leichte Aufgabe, da er sich in den Quellcode des Programmierers einarbeiten und diesen testen muss. 
Damit ist es aber noch nicht getan, da der Tester nun den Fehler weiter geben muss.
Das absichtliche Suchen nach Fehlern und dem Drang nach Perfektionismus durchzusetzen ist zwar notwendig aber kann schnell zu zwischenmenschlichen Konflikten führen. 
Deshalb ist sowohl auf Tester- und Programmierseite Vorsicht und auch Verständnis für der Aufgabe des Gegenparts geboten.