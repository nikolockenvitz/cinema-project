% !TEX root =  master.tex
\section{Technische Grundlagen der Tests}

\subsection{JUnit}

\subsubsection{Allgemeines}

JUnit ist ein Java-Framework, welches vorwiegend zum automatisieren von Tests verschiedener Klassen und Methoden verwendet wird. \footnote{\url{https://junit.org/junit5/docs/current/user-guide/}}

Ein Test kann bei JUnit zwei Ergebnisse erzielen, entweder er schlägt fehl und wird mit rot markiert oder er läuft erfolgreich ab und wird grün markiert.
Bei den fehlgeschlagenen Tests wird jedoch noch weiter unterschieden. 
Es gibt sogenannte \enquote{Failures}, welche dadurch charakterisiert werden, dass nicht das erwartete Ergebnis beim Test aufgetreten ist.
Des Weiteren gibt es sogenannte \enquote{Error}, welche unerwartete Fehler sind, die während eines Tests auftreten können und somit den Test möglicherweise nicht vollenden lassen oder ein falsches Ergebnis hervorrufen.

Tests werden zudem als eigene Klassen realisiert, um sie vom Programmcode abzugrenzen und nicht vorm Kompilieren des Projektes entfernen zu müssen.

\subsubsection{Funktionsweise}

JUnit-Tests werden in Java mit Annotationen angekündigt und können mit diesen spezialisiert werden.
Standardmäßig wird ein Test mit \enquote{@Test} eingeleitet, danach folgt die eigentliche Test-Methode.
In der Test-Methode werden die zu testenden Klassen bzw. Methoden aufgerufen und mit Hilfe der \textit{assertEquals}-Methode getestet.
Hierbei wird ein erwarteter Wert bzw. ein erwartetes Objekt angegeben und diese mit den Ergebnissen der zu testenden Methoden auf Gleichheit geprüft.

Falls mehere Tests die selben Ausgangsdaten benötigen, kann mit Hilfe der \enquote{@BeforeAll}-Annotation einmalig zu Beginn bzw. mit \enquote{@BeforeEach} vor jeder Test-Methode eine Initialisierungsmethode eingeleitet werden, um die Ausgangsdaten zu erzeugen.
Gleichzeitig gibt es diese Annotationen auch nach den Tests, um bspw. Datenbankverbindungen erneut zu schließen. Diese Methoden werden mit \enquote{@AfterAll} und \enquote{@AfterEach} eingeleitet.

Außerdem können Tests überspringen werden, sofern sie mit \enquote{@Disabled} eingeleitet werden.
Dies hat einerseits Vorteile bei der Entwicklung, da so Tests schneller geprüft werden können, andererseits können so Tests auskommentiert werden sofern Bugs in den zu testenden Methoden vorhanden sind.


\subsection{Hamcrest}

Hamcrest ist wie JUnit ein Java-Framework, welches es leichter macht richtige Matcher-Objekte zu deklarieren. Jedoch ist Hamcrest als Zusatz zu JUnit zu betrachten und nicht als eigenständiges Test-Framework.
Dies ist ein großer Vorteil gegenüber dem Standard-Matcher von JUnit, da dieser allein mit der Gleichheit der Objekte arbeitet.
Somit is es einfacher komplette Kollektionen von Objekten auf Gleichheit zu testen oder komplexere Tests aufzustellen.

Hamcrest erlaubt es dem Nutzer zusätzlich auch eigene Matcher zu erstellen, welche das Testen erheblich erleichtern können.

