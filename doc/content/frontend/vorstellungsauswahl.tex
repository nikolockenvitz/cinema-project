% !TEX root =  master.tex
\section{Filmdetails und Vorstellungsauswahl}

Nachdem die Daten wie in Quelltext \ref{lst:load_movie_and_shows} geladen wurden, muss nun das \acs{DOM} entsprechend verändert werden.
Die statische \acs{HTML}-Seite ist dabei auf das Wesentliche reduziert und enthält neben der Menüleiste und der Fußzeile folgenden Teil:

\begin{lstlisting}[style=lstHTML]
<div class="container" id="movie">
	<!-- Informationen zum Film werden hier eingefügt -->
	<table class="table" id="shows">
		<tr>
			<th>Tag</th>
			<th>Vorführungen</th>
		</tr>
		<!-- die einzelnen Vorstellungen werden hier eingefügt -->
	</table>
	<!-- Bewertungen werden hier eingefügt -->
</div>
\end{lstlisting}
\captionof{lstlisting}{\acs{HTML}-Seite für Filmdetails}
\label{lst:html_movie_and_shows}

An den gekennzeichneten Stellen wird später der Inhalt, der aus dem Back-End geladen wurde, eingefügt.
Dafür gibt es kleinere Vorlagen mit Platzhaltern.
In Quelltext \ref{lst:html_template_movie_detail} sieht man die Vorlage für die Filmdetails.

\begin{lstlisting}[style=lstHTML,escapeinside=``]
<div class="row featurette">
	<div class="col-12 col-sm-4">
		<img class="featurette-image img-fluid mx-auto" src="./img/`\textcolor{red}{\{movieID\}}`.jpg" alt="`\textcolor{red}{\{movieTitle\}}`" width="100%">
	</div>
	<div class="col-12 col-sm-8">
		<h2 class="featurette-heading">`\textcolor{red}{\{movieTitle\}}`</h2>
		<p class="lead">FSK `\textcolor{red}{\{movieFSK\}}`, `\textcolor{red}{\{movieDuration\}}` Minuten, `\textcolor{red}{\{movieGenres\}}`</p>
		<p class="lead rating-`\textcolor{red}{\{movieRatingRounded\}}`">`\textcolor{red}{\{movieRating\}}`</p>
	</div>
</div>

<div class="row featurette">
	<div class="col-12">
		<p class="lead">`\textcolor{red}{\{movieDescription\}}`</p>
	</div>
</div>
\end{lstlisting}
\captionof{lstlisting}{\acs{HTML}-Vorlage mit Platzhaltern für Filmdetails}
\label{lst:html_template_movie_detail}

In JavaScript wird diese Vorlage dann mit Inhalten befüllt.
Nachdem die Daten in der Funktion \textit{displayMovieAndShows()} ein wenig verarbeitet wurden, können sie mit JavaScript bzw. jQuery in die Vorlage und danach ins \acs{DOM} eingefügt werden.

\begin{lstlisting}[language=JavaScript]
$("#movie").prepend(templateMovieDetail
	.replace("{movieID}", movie.id)
	.replace(/\{movieTitle\}/g, movie.name)
	.replace("{movieFSK}", movie.fsk)
	.replace("{movieDuration}", movie.duration)
	.replace("{movieGenres}", genres)
	.replace("{movieRatingRounded}", Math.max(0, Math.min(5, Math.round(rating))))
	.replace("{movieRating}", (movie.ratings.length > 0 ? rating.toFixed(1).replace(".",",") : "") + " (" + movie.ratings.length + " Bewertung" + (movie.ratings.length == 1 ? "" : "en") + ")")
	.replace("{movieDescription}", movie.description)
);
\end{lstlisting}
\captionof{lstlisting}{Einfügen der Filmdetails ins \acs{DOM} mit jQuery}
\label{lst:js_write_movie_detail_to_dom}

Genauso wie die Filmdetails werden auch die Vorstellungen und die Bewertungen ins \acs{DOM} eingefügt.

Zusammenfassend heißt das, dass zuerst einmal die statische und weitestgehend leere \acs{HTML}-Seite geladen wird.
Diese wiederum importiert eine JavaScript-Datei, welche beim Laden der Seite einen Aufruf an das Back-End schickt.
Sobald der Server antwortet, wird die Antwort entsprechend verarbeitet und in JavaScript in kleinere Vorlagen eingefügt.
Die mit Inhalten befüllten Vorlagen werden dann mit jQuery ins \acs{DOM} eingefügt, sodass die \acs{HTML}-Seite die gewünschten Inhalte darstellt.
Dies passiert im Idealfall so schnell, dass es für den Benutzer wirkt, als hätte er die Seite \enquote{ganz normal} geladen.

Damit dies reibungslos funktioniert, ist es essentiell, dass sich die Kommunikation mit dem Server auf ein Minimum beschränkt und keine nicht benötigten oder redundanten Daten transferiert werden.
So ist es zum Beispiel schlecht für die Performance, wenn man im Front-End lediglich eine Liste mit allen Filmen sehen will, aber vom Back-End zu jedem Film alle Vorstellungen mit allen Sitzplätzen erhält.

Genauso muss natürlich die Verarbeitung im Front-End effizient erfolgen.