% !TEX root =  master.tex
\section{Anbindung an das Back-End}
\label{sec:anbindung_backend}
\authorsection{\authorNL}

Um die Daten aus der Datenbank bzw. dem Back-End anzuzeigen, werden diese durch einen oder mehrere \acs{AJAX}-Aufrufe nachgeladen.
Beim Laden der Startseite wird die Funktion \textit{loadMovies()} aufgerufen.
Diese wiederum ist nur für die Zuordnung von einem Pfad bzw. einer \acs{URI} zu einer Funktion, die das Ergebnis verarbeitet, da.
Sie würde auch \acs{URL}-Parameter auslesen und in den Aufruf mit einarbeiten, dies ist aber auf der Startseite noch nicht notwendig.

\begin{lstlisting}[language=JavaScript, caption={\acs{AJAX}-Aufruf, um alle Filme zu erhalten}, label={lst:ajax_all_movies}]
const URL_SERVER = "http://localhost:8080/cinema-system";
const PATH_ALL_MOVIES = "/movie/getAllMovies";

function getData (path, func) {
	$.ajax({
		type: "GET",
		url: URL_SERVER + path,
		success: (data) => func(data)
	});
}

function loadMovies () {
	getData(PATH_ALL_MOVIES, displayMovies);
}

function displayMovies (movies) {
	...
}
\end{lstlisting}

Der \acs{AJAX}-Aufruf wird asynchron durchgeführt.
Das Ergebnis wird, wenn es beim Client angekommen ist, an die Funktion \textit{displayMovies()} weitergegeben, die nun dafür zuständig ist, die Daten entsprechend anzuzeigen.

Die Funktion \textit{getData()} dient hier als Schnittstelle zum Back-End und wird im Grunde an allen Stellen genutzt, bei denen Daten vom Back-End abgefragt werden.
Als Parameter erhält sie den Pfad, der vom Server nachgefragt werden soll, sowie eine Referenz auf eine Funktion, die das Ergebnis verarbeitet.

Wenn der Aufruf an das Back-End noch einen oder mehrere Parameter enthält, wie zum Beispiel beim Laden eines einzelnen Films mit seinen Details und Vorstellungen, so wird dieser Parameter in der zugehörigen \textit{load}-Funktion hinzugefügt.

\begin{lstlisting}[language=JavaScript, caption={Laden der Filmdetails nach Auslesen der \acs{URL}-Parameter}, label={lst:load_movie_and_shows}]
const PATH_MOVIE = "/movie/{movieID}";

function loadMovieAndShows () {
	var movieID = urlparameters.get("id");
	if (movieID == null) {
		window.location.href = "./index.html";
	}
	else {
		getData(PATH_MOVIE.replace("{movieID}", movieID), displayMovieAndShows);
	}
}

function displayMovieAndShows (movie) {
	...
}
\end{lstlisting}

Wie in Quelltext \vref{lst:load_movie_and_shows} erkennbar ist, wird der Platzhalter \textit{movieID} einfach durch den aus der \acs{URL} ausgelesenen Wert ersetzt.
Danach wird genauso wie in Quelltext \vref{lst:ajax_all_movies} die Anfrage an das Back-End gestellt und das Ergebnis in einer weiteren Funktion verarbeitet und angezeigt.