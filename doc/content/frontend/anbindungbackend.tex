% !TEX root =  master.tex
\section{Anbindung an das Back-End}
\label{sec:anbindung_backend}

Um die Daten aus der Datenbank bzw. dem Back-End anzuzeigen, werden diese durch einen oder mehrere \acs{AJAX}-Aufrufe nachgeladen.
Beim Laden der Startseite wird die Funktion \textit{loadMovies()} aufgerufen.
Diese wiederum ist nur für die Zuordnung von einem Pfad bzw. einer \acs{URI} zu einer Funktion, die das Ergebnis verarbeitet, da.
Sie würde auch \acs{URL}-Parameter auslesen und in den Aufruf mit einarbeiten, dies ist aber auf der Startseite noch nicht notwendig.

\begin{lstlisting}[language=JavaScript]
const URL_SERVER = "http://localhost:8080/cinema-system";
const PATH_ALL_MOVIES = "/movie/getAllMovies";

function getData (path, func) {
	$.ajax({
		type: "GET",
		url: URL_SERVER + path,
		success: (data) => func(data),
		error: function (xhr,status,error){
			console.log(xhr, status, error);
			func([]);
		}
	});
}

function loadMovies () {
	getData(PATH_ALL_MOVIES, displayMovies);
}

function displayMovies (movies) {
	...
}
\end{lstlisting}
\captionof{lstlisting}{\acs{AJAX}-Aufruf, um alle Filme zu erhalten}
\label{lst:ajax_all_movies}