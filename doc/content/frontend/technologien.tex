% !TEX root =  master.tex
\section{Verwendete Technologien und Konzept}

Bei der Implementierung des Front-Ends fiel die Wahl auf die Verwendung von \acs{HTML} und \acs{CSS} in Verbindung mit dem Framework Bootstrap.
So lässt sich mit einfachen Mitteln eine ansprechende und responsive Benutzeroberfläche gestalten.

Da der Inhalt der Benutzeroberfläche dynamisch sein soll, müssen irgendwann die Daten aus dem Back-End mit dem statischen Inhalt verknüpft werden.
Dafür gibt es im Grunde zwei Möglichkeiten.
Entweder man generiert die anzuzeigende Webseite vollständige im Back-End oder man hat statische Seiten und lädt die Inhalte im Anschluss nach.

Es wurde die zweite Lösung gewählt, da sie eine bessere Trennung der verschiedenen Schichten bietet und zudem auch schneller ist.
So wird das Einfügen der dynamischen Inhalte clientseitig ausgeführt und belastet den Server entsprechend weniger.
Ebenso bietet es den Vorteil, dass bei einer Änderung nicht jedes Mal die Seite neu geladen werden muss.
Dies ist bereits jetzt beim Blockieren von Sitzplätzen zu finden, andererseits kann man das Front-End so auch noch später auf eine Single-Page-Webanwendung umstellen.

Für die Anpassung der \acs{HTML}-Seite, genauer gesagt des \acs{DOM}s im Browser, wird Java\-Script mit der Bibliothek jQuery genutzt.
Durch sogenannte \acs{AJAX}-Aufrufe werden Daten wie zum Beispiel eine Liste mit allen Filmen vom Back-End geladen.
Das zurückgelieferte \acs{JSON} wird in JavaScript dann verarbeitet und die Daten daraus werden in kleine \acs{HTML}-Vorlagen eingefügt.
Nachdem die \acs{HTML}-Vorlagen mit Daten befüllt wurden, können sie im \acs{DOM} platziert werden.
